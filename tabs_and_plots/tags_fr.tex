\begin{table}[ht]
\centering
\scalebox{0.6}{\begin{tabular}{lll}
\toprule
Tag & Tag name (in singular) & Description \\
\midrule
\midrule
PER & \makecell{de noms de personnes (un nom \\de personne) } & \makecell{Il s'agit des noms de personnes, qu'elles soient réelles ou fictives. } \\
\midrule
FAC & \makecell{de productions humaines (une \\production humaine) } & \makecell{Il s'agit des noms de structures faites par les humains comme des infrastructures, des bâtiments ou des monuments. } \\
\midrule
LOC & \makecell{de lieux (un lieu) } & \makecell{Il s'agit des noms de lieux comme des endroits, villes, pays ou régions. } \\
\midrule
ORG & \makecell{d'organisations (une \\organisation) } & \makecell{Il s'agit des noms d'organisations comme des entreprises, des agences ou des partis politiques. } \\
\midrule
FUNC & \makecell{de fonctions et métiers (une \\fonction ou un métier) } & \makecell{Il s'agit de mots qui se rapportent à une activité professionnelle. } \\
\midrule
ANAT & \makecell{d'anatomie (une partie du \\corps) } & \makecell{Il s'agit d'une entité se rapportant à la structure du corps humain, ses organes et leur position. Il s’agit \\principalement des parties du corpus ou organes, des appareils, des tissus, des cellules, des substances corporelles et \\des organismes embryonaires. } \\
\midrule
CHEM & \makecell{de médicaments et substances \\chimiques (un médicament ou \\une substance chimique) } & \makecell{Il s'agit d'une substance ou composition présentant des propriétés chimiques caractéristiques, en particulier des \\propriétés curatives ou préventives à l’égard des maladies humaines ou animales. Il s’agit principalement des \\médicaments disponibles en pharmacie, des antibiotiques, des proteines, des hormones, des substances dangereuses, des \\enzymes. } \\
\midrule
DEVI & \makecell{de matériel (un matériel) } & \makecell{Il s'agit d'un matériel utilisé pour administrer des soins ou effectuer des recherches médicales. } \\
\midrule
DISO & \makecell{de problèmes médicaux (un \\problème médical) } & \makecell{Il s'agit d'une altération de la morphologie, des fonctions, ou de la santé d’un organisme vivant, animal ou végétal. Il \\peut s’agir de malformations, de maladies, de blessure, de signe ou symptome ou d’une observation. } \\
\midrule
GEOG & \makecell{de zones géographiques (une \\zone géographique) } & \makecell{Il s'agit d'un pays, une région, ou une ville. } \\
\midrule
LIVB & \makecell{d'êtres vivants (un être \\vivant) } & \makecell{Il s'agit d'un être vivant ou groupe d’êtres vivants. Il peut s’agir d’une personne ou d’un groupe de personnes, d’une \\plante ou d’une catégorie de végétaux, d’un animal ou d’une catégorie d’animaux. } \\
\midrule
OBJC & \makecell{d'objets (un objet) } & \makecell{Il s'agit de tout ce qui, animé ou inanimé, affecte les sens. Ici, il s’agit principalement d’objets physiques \\manufacturés. } \\
\midrule
PHEN & \makecell{de phénomènes (un phénomène) } & \makecell{Il s'agit d'un phénomène qui se produit naturellement ou à la suite d’une activité. Il s’agit principalement de \\fonctions biologiques. } \\
\midrule
PHYS & \makecell{de physiologie (une \\physiologie) } & \makecell{Il s'agit de tout élément contribuant au fonctionnement ou à l’organisation mécanique, physique et biochimique des \\organismes vivants et de leurs composants. } \\
\midrule
PROC & \makecell{de procédures (une procédure) } & \makecell{Il s'agit d'une activité ou procédure contribuant au diagnostic ou au traitement des patients, à l’information des \\patients, la formation du personnel médical ou à la recherche biomédicale. } \\
\midrule
EVENT & \makecell{d'événements (un événement) } & \makecell{Il s'agit d'une action, d’un état ou d’une circonstance qui est pertinent pour l’histoire clinique d’un patient. Il peut \\s’agir de pathologies et symptômes, ou plus généralement de mots comme "entre", "rapporte" ou "continue". } \\
\midrule
TIMEX3 & \makecell{d'expressions temporelles (une \\expression temporelle) } & \makecell{Il s'agit d’expressions temporelles comme des dates, heures, durées, fréquences, ou intervalles. } \\
\midrule
RML & \makecell{de résultats et mesures (un \\résultat ou une mesure) } & \makecell{Il s'agit de résultats d’analyses de laboratoire, de mesures formelles, et de valeurs de mesure. } \\
\midrule
ACTOR & \makecell{d'acteurs (un acteur) } & \makecell{Il s'agit de patients, de professionnels de santé, ou d’autres acteurs pertinents pour l’histoire clinique d’un patient. } \\
\midrule
\bottomrule
\end{tabular}}
\caption{Description of the NER tags used in our experiments for French.}
\label{tab:ner_tags_fr}
\end{table}
